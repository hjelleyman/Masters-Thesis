\section{Variability of expected sea ice extent}
Considering the anomalous seasonal prediction, we can evaluate how good our model is with respect to the interannyal variability by plotting the expected sea ice extent against the true sea ice extent. In this case we ignore the time axis and plot the two ice extents together. We get the below plot.

Here wee can see that the model only accounts for about 5\% of the variability of sea ice extent in Antarctica with only a 0.22 pearson correlation coefficient between the two time series. This is unsurprising because the model is only fitting linearly and the system is much more complex and we expect non-linear relationships between global climate and the behaviour of sea ice in Antarctica. We do however find that the model fits the total sea ice extent with statistical significance, (a p-value of 0.01) Consequently we can see that while the model doesn't account for a large proportion of variability, the variability it expresses still contains useful information.

Considering the anomalous annual sea ice extent prediction, we notice that the gradient has increased to 5\% of variability being explained by our model. This also has a higher pearson correlation coefficient of 0.28, however the relationship isn't statistically significant with a p-value of 0.09. This will be due to the smaller number of data points. 