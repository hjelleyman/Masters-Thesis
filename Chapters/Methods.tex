\chapter{Methods}
\label{Chap:Methods}

\section{Introduction to Methods}
This chapter includes a description of the methods employed in this project. We will cover everything from standardising the datasets for easy comparison through to \todo{finish this once the section is done}

\section{Regridding Data}
Because we use a variety of datasets which come in a variety of structures, it is important that standardise the spatial dimensions of each data source. One way we do this is by interpolating each dataset to have a consistent spatial arrangement. This allows for better quality results and makes it easier to calculate measures such as the correlation between 2m temperature and sea ice concentration.

We do the interpolation using the python package Scipy, which makes use of a piecewise cubic, continuously differentiable (C1), and approximately curvature-minimizing polynomial surface to determine the value of our given variable at a chosen location. \todo{cite this or something?}

We converted the temperature data to the projection the sea ice data is provided in; a south polar stereographic projection with regular grid cells of 25km$\times$25km. We found this resolution to have a good balance between reasonable runtimes and good quality results.


\section{Pearson Correlation Coefficient}
\label{Methods:pearson}
\todo{Do we need a citation for this?}
One comparison which we used for identifying connections between two different time series or different time series components is the Pearson correlation component. Taking a time series $x$, and a time series $y$, with elements $x_i$ and $y_i$, and means denoted by $\bar{x}$ and $\bar{y}$; we can calculate the Pearson correlation coefficient $r_{xy}$.

\begin{equation}
\label{eq:pearson}
    r(x, y)=\frac{\sum_{i=1}^{n}\left(x_{i}-\overline{x}\right)\left(y_{i}-\overline{y}\right)}{\sqrt{\sum_{i=1}^{n}\left(x_{i}-\overline{x}\right)^{2}} \sqrt{\sum_{i=1}^{n}\left(y_{i}-\overline{y}\right)^{2}}}
\end{equation}
