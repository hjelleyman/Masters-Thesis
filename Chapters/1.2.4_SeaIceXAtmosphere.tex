\documentclass[../main.tex]{subfiles}
\begin{document}
\section{Sea ice and atmospheric processes}
This section of the literature review will discuss in some detail, the research which has been carried out to investigate the relationship between sea ice and atmospheric processes in Antarctica and globally.

Comiso et al look with some detail at the relationship between sea ice in Antarctica and surface temperature \cite{ComisoPositiveTemperature}. While their dataset ends in 2015, just before the amount of ice in Antarctica experiences a rapid decrease in concentration, their research intent and their methods are still relevant to us as we look at sea ice in Antarctica today. They provide a good commentary on the quality of the satellite measurements from different sensors, before moving into a correlation analysis, looking at the relationship between surface temperature and sea ice in Antarctica. One thing they do is to break the continent into sections and look at how the correlation changes in each region. Likewise they break the temporal scale into seasons while keeping the overall picture. When looking for other environmental influences they found smaller than expected correlations with patterns like the Southern Annular Mode (SAM), however they speculate that ENSO may be a major contributing factor to the patterns in sea ice in Antarctica. On the whole, they found a positive correlation between the SIE in Antarctica and surface temperature in the same region, with an even larger correlation when you introduce a time-lag. 

In 2016 we saw a record low in Antarctic Sea ice extent. Wang et al \cite{Wang2019Compounding2016} discuss some of the physical processes which could be a cause of this extreme event. Their results indicate to them that this was largely due to naturally occurring variability, nonetheless they are unable to discount a possible role of anthropogenic forcing. They link the extreme concentrations of ice to a anomalous atmospheric circulation over the Indian and western Pacific oceans and unusual internal atmosphere-ocean variability. Of interest to us here are the different atmospheric circulation indices they argue has an impact on the patterns of sea ice concentration in Antarctica. The look at the Indian Ocean Dipole (IOD), Madden Julian Oscillation (MJO), ENSO and SAM, as contributors to this event.


\cite{Meehl2019Sustained2016} wrote one of the key papers for our project so we will use it as a starting point. And explore the literature using its references which seem relevant to this topic. The paper focuses on the impacts on the sudden sea ice retreat in 2016 where we saw record low SIE. The first detail of point is that they classify the sections of their SIE time series by when IPO is positive and negative with a 13 year low-pass Lanczos filter. They associate IPO with acceleration or slowdown of global warming, thus relating it to long term trends in sea ice. (Acceleration in positive phase and slowing in negative phase) They say that the acceleration off antarctic sea ice growth was predominantly driven by negative convective heating anomalies in the tropical Pacific.

They also discuss the sudden decrease seen in 2016. They claim this occurred because of a zonal wave number 3 pattern enhancing meridional flow, and negative SAM values towards the end of the year. The DMI index caused positive SST anomalies in the tropical eastern Indian Ocean and the far-western Pacific. This enhanced convection during SON and indicated by record low OLR for the area 90 E - 150 E and an associated precipitation anomaly.

The main takeaway can be drawn from the end of their introduction; First, teleconnections from strong tropical convection in the eastern Indian Ocean produced surface wind anomalies.
They say that a negative phase of IPO and positive phase of SAM associated with strengthened westerlies moved warm subsurface water upwards due to Ekmann suction (On the long term). Third, a negative phase of SAM and transition to a positive IPO produced warm SSTs to complete a warming of the upper 600m of the ocean.
\medskip

\cite{Doddridge2017ModulationMode} discuss in some detail, some of the impact of SAM on the Antarctic SIE seasonal cycle. Their primary finding is that positive SAM anomalies in summer result in cold SST and anomalous ice growth in the following summer, while negative anomalies in SAM can be associated with a reduction in SIE on the following Autumn. The increase in SAM is notably largest in summer and has been linked to the depletion of stratospheric ozone over Antarctica. They note that other papers have found some evidence of the SAM affecting Antarctic SIE in the Indian Ocean during MJJ, but that it is not well explained by SAM. They mention a two time scale response which can be used to explain this relationship. This is explored by \cite{FerreiraAntarcticProblem} in some detail and may be related to our research. \cite{Doddridge2017ModulationMode} use composites and regression analysis to look at the relationship between SAM and SIE on a seasonal basis finding the results described above. They find the same signal in the raw and detrended datasets.


\cite{Simpkins2012SeasonalConcentration} and \cite{Kohyama2016AntarcticVariability} discuss with some technical analysis the impact of long term behaviours between Antarctic SIC with ENSO and SAM.



\cite{Clem2020RecordDecades} establish a link between tropical modes of atmospheric climate and surface air temperature (SAT) over the internal Antarctic region. While this isn't directly Sea ice, it is good physical evidence that these relationships exist which supports our hypothesis that SAM impacts the long term variability of Antarctic SIE.

\cite{Turner2020a} Detail some physical reasons for the decrease in Antarctic SIE in 2016.
\end{document}